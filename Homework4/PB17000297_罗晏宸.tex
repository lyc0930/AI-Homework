\documentclass{article}
\usepackage[UTF8]{ctex}
\usepackage[T1]{fontenc}
\usepackage[utf8]{inputenc}
\usepackage{titlesec}
\usepackage[colorlinks, linkcolor = black]{hyperref}
\usepackage{float}
\usepackage{xcolor}
\usepackage{amsmath}
\usepackage{amssymb}
\usepackage{latexsym}
\usepackage{amsthm}
\usepackage{graphicx}

\title{Homework 4}
\author{PB17000297 罗晏宸}
\date{March 22 2020}

\begin{document}
\maketitle

\section{Exercise 7.13}
本题考虑子句和蕴涵语句之间的关系。
\subparagraph{a}
证明子句$(\lnot P_1 \lor \cdots \lor \lnot P_m \lor Q)$逻辑等价于蕴涵语句$(P_1 \land \cdots \land P_m) \Rightarrow Q$。
\subparagraph{b}
证明每个子句(不管正文字的数量)都可以写成$(P_1 \land \cdots \land P_m) \Rightarrow (Q_1 \lor \cdots \lor Q_n)$的形式,其中$P_i$和$Q_i$都是命题词。由这类语句构成的知识库是表示为\textbf{蕴涵范式}或称\textbf{Kowalski 范式}(Kowalski, 1979)。
\subparagraph{c}
写出蕴涵范式语句的完整归结规则。

\paragraph{解}
\subparagraph{a}
\begin{proof}
    由De Morgan律,有
    \begin{equation*}
        (\lnot P_1 \lor \cdots \lor \lnot P_m \lor Q) \equiv (\lnot (P_1 \land \cdots \land P_m) \lor Q)
    \end{equation*}
    由蕴涵消去,有
    \begin{equation*}
        (\lnot (P_1 \land \cdots \land P_m) \lor Q) \equiv ((P_1 \land \cdots \land P_m) \Rightarrow Q)
    \end{equation*}
\end{proof}

\subparagraph{b}
\begin{proof}
对于一个子句,假设其中有$n$个正文字,分别为$Q_1,\, Q_2,\, \cdots,\, Q_n$,有$m$个负文字,分别为$\lnot P_1,\, \lnot P_2,\, \cdots,\, \lnot P_m$,则这个子句可以表示为$$\lnot P_1 \lor \cdots \lor \lnot P_m \lor Q_1 \lor \cdots \lor Q_n$$,由上可知
\begin{equation*}
    (\lnot P_1 \lor \cdots \lor \lnot P_m \lor Q_1 \lor \cdots \lor Q_n) \equiv ((P_1 \land \cdots \land P_m) \Rightarrow (Q_1 \lor \cdots \lor Q_n))
\end{equation*}
\end{proof}

\subparagraph{c}
应用于蕴涵范式的全归结规则如下:
\begin{equation*}
\resizebox{\linewidth}{!}{%
    $\displaystyle \frac{(p_1 \land \cdots \land p_m) \Rightarrow (q_1 \lor \cdots \lor q_n), \qquad (r_1 \land \cdots \land r_l) \Rightarrow (s_1 \lor \cdots \lor s_k)}{(p_1 \land \cdots \land p_{i - 1} \land p_{i + 1} \land \cdots \land p_m \land r_1 \land \cdots \land r_l) \Rightarrow (q_1 \lor \cdots \lor q_n \lor s_1 \lor \cdots \lor s_{j - 1} \lor s_{j + 1} \lor \cdots \lor s_k)}$
}
\end{equation*}
其中每一个$p$,$q$,$r$,$s$都是文字,且$p_i = s_j$。
\section{Supplementary Exercise}
证明Forward Chaining algorithm的完备性

\paragraph{解}
\begin{proof}
    前向链接是完备的即每个被蕴涵的原子语句都可以推导得出。考察\textit{inferred}表的最终状态(在算法到达不动点以后,不会再出现新的推理)。该表把推导出的每个符号设为\textit{true},而其他符号为\textit{false}。可以把此表看作一个逻辑模型;而且,原始\textit{KB}中的每个限定子句在该模型中都为真。假设相反的情况成立,即某个子句$a_1 \land \cdots \land a_k \Rightarrow b$在此模型下为假。那么$a_1 \land \cdots \land a_k$在模型中必须为真,$b$必须为假。但这与算法已到达一个不动点的假设矛盾。因此在不动点推导出的原子语句集定义了原始\textit{KB}的一个模型。更进一步,被\textit{KB}蕴涵的任一原子语句$q$在它的所有模型中为真,尤其是这个模型。因此每个被蕴涵的语句$q$都可以被算法推导得出。
\end{proof}

\end{document}