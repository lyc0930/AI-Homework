\documentclass{article}
\usepackage[UTF8]{ctex}
\usepackage[T1]{fontenc}
\usepackage[utf8]{inputenc}

\title{Homework 1}
\author{PB17000297 罗晏宸}
\date{March 1 2020}

\begin{document}
\maketitle

\section{Exercise 3.6}
对以下问题给出完整的形式化。选择的形式化方法要足够精确以便于实现。
\subparagraph{a}
只用四种颜色对平面地图着色,要求每两个相邻的地区不能具有相同的颜色。
\subparagraph{b}
屋子里有只3英尺高的猴子,离地8英尺的屋顶上挂着一串香蕉。猴子想吃香蕉。屋子里有两个可叠放、可移动、可攀爬的3英尺高的箱子。
\subparagraph{d}
有三个水壶,容量分别为12加仑、8加仑和3加仑,还有一个放液嘴。可以把水壶装满或者倒空,从一个壶倒进另一个壶或者倒在地上。请量出刚好1加仑水。

\paragraph{解}
\subparagraph{a}
\subparagraph{b}
\subparagraph{d}

\section{Exercise 3.9}
\textbf{传教士和野人}问题。三个传教士和三个野人在河的一岸,有一条能载一个人或者两个人的船。请设法使所有人都渡到河的另一岸,要求在任何地方野人数都不能多于传教士的人数,这个问题在AI领域中很有名,是因为它是第一个从分析的观点探讨问题形式化的论文的主题(Amarel, 1968)。
\subparagraph{a}
请对该问题进行详细形式化,只描述确保该问题求解所必需的特性。画出完整的状态空间图。
\subparagraph{b}
应用合适的搜索算法求出该问题的最优解。对于这个问题检查重复状态是个好主意吗?
\subparagraph{c}
这个问题的状态空间很简单,你认为是什么导致人们求解它很困难?


\end{document}