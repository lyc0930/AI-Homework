\documentclass{article}
\usepackage[UTF8]{ctex}
\usepackage[T1]{fontenc}
\usepackage[utf8]{inputenc}

\title{Homework 1}
\author{PB17000297 罗晏宸}
\date{March 1 2020}

\begin{document}
\maketitle

\section{Exercise 3.6}
对以下问题给出完整的形式化。选择的形式化方法要足够精确以便于实现。
\subparagraph{a}
只用四种颜色对平面地图着色,要求每两个相邻的地区不能具有相同的颜色。
\subparagraph{b}
屋子里有只3英尺高的猴子,离地8英尺的屋顶上挂着一串香蕉。猴子想吃香蕉。屋子里有两个可叠放、可移动、可攀爬的3英尺高的箱子。
\subparagraph{d}
有三个水壶,容量分别为12加仑、8加仑和3加仑,还有一个放液嘴。可以把水壶装满或者倒空,从一个壶倒进另一个壶或者倒在地上。请量出刚好1加仑水。

\paragraph{解}
\subparagraph{a}
\begin{itemize}
    \item 初始状态:所有地区均未被着色
    \item 可能的行动:给定一个状态,对其中一个未被着色的地区分配一种和与其相邻的已着色地区都不相同的颜色
    \item 转移模型:从一个状态达到一个已着色地区数加一的状态
    \item 目标测试:所有地区都已被着色,且每两个相邻的地区都不具有相同的颜色
    \item 路径耗散:总共的着色次数
\end{itemize}
\subparagraph{b}
\begin{itemize}
    \item 初始状态:箱子未被叠放,猴子活动在地面上
    \item 可能的行动:在(某一高度上)移动,移动一个(猴子不站立在其上的)箱子,(在两个箱子均放置在地面上时)叠放两个箱子,跳上一个(与猴子在同一高度的)箱子,从一个箱子上跳下,(在猴子最高处达到或超过香蕉高度时)摘取香蕉
    \item 转移模型:从一个状态达到另一个状态,猴子以及两个箱子彼此相对位置改变
    \item 目标测试:猴子摘取到香蕉
    \item 路径耗散:行动次数
\end{itemize}
\subparagraph{d}
\begin{itemize}
    \item 初始状态:三个水壶均为空,记为三元组$(0,\,0,\,0)$
    \item 可能的行动:把其中一个水壶倒满,把其中一个水壶倒空,从一个壶倒进另一个壶
    \item 转移模型:对于状态$(x_1,\,x_2,\,x_3)$,把一个水壶倒满后可以达到$(3,\,x_2,\,x_3)$、$(x_1,\,8,\,x_3)$或$(x_1,\,x_2,\,12)$,把一个水壶倒空后可以达到$(0,\,x_2,\,x_3)$、$(x_1,\,0,\,x_3)$或$(x_1,\,x_2,\,0)$,从一个壶$i(i = 1,2,3)$倒进另一个壶$j(j = 1,2,3,\, j \neq i)$后可以达到 $x'_i = 0,\, x'_j = x_i + x_j,\ (x_i + x_j \leq C_j)$ 或 $x'_i = x_i + x_j - C_j,\, x'_j = C_j,\ (x_i + x_j > C_j)$ ,其中$C_j$是壶$j$的容量
    \item 目标测试:其中一个水壶中刚好有一加仑水
    \item 路径耗散:行动次数
\end{itemize}

\section{Exercise 3.9}
\textbf{传教士和野人}问题。三个传教士和三个野人在河的一岸,有一条能载一个人或者两个人的船。请设法使所有人都渡到河的另一岸,要求在任何地方野人数都不能多于传教士的人数,这个问题在AI领域中很有名,是因为它是第一个从分析的观点探讨问题形式化的论文的主题(Amarel, 1968)。
\subparagraph{a}
请对该问题进行详细形式化,只描述确保该问题求解所必需的特性。画出完整的状态空间图。
\subparagraph{b}
应用合适的搜索算法求出该问题的最优解。对于这个问题检查重复状态是个好主意吗?
\subparagraph{c}
这个问题的状态空间很简单,你认为是什么导致人们求解它很困难?


\end{document}